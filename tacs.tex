\documentclass[9pt,twocolumn,twoside]{pnas-new}

\templatetype{pnasresearcharticle} % Choose template 

\title{Sub-structure within Th2 and non-Th2 molecular phenotypes of asthma using sputum transcriptomics in U-BIOPRED }

% Use letters for affiliations, numbers to show equal authorship (if applicable) and to indicate the corresponding author
\author[a,c,1]{Author One}
\author[b,1,2]{Author Two} 
\author[a]{Author Three}

\affil[a]{Affiliation One}
\affil[b]{Affiliation Two}
\affil[c]{Affiliation Three}

% Please give the surname of the lead author for the running footer
\leadauthor{Lead author last name} 


\equalauthors{\textsuperscript{1}A.O.(Author One) and A.T. (Author Two) contributed equally to this work (remove if not applicable).}
\correspondingauthor{\textsuperscript{2}To whom correspondence should be addressed. E-mail: author.two\@email.com}

\keywords{asthma $|$ clustering 2 $|$ sub-typing 3 $|$ ...} 

\begin{abstract}
Please provide an abstract of no more than 250 words in a single paragraph. Abstracts should explain to the general reader the major contributions of the article. References in the abstract must be cited in full within the abstract itself and cited in the text.
\end{abstract}

\dates{This manuscript was compiled on \today}
\doi{\url{www.pnas.org/cgi/doi/10.1073/pnas.XXXXXXXXXX}}

\begin{document}

\maketitle

\section*{Introduction}

Precision medicine is often defined as being "the right treatment, for the right patient at the the right time". By sub-typing patients, breaking apparently homogeneous groups based on clinical presentation down into groups unified by shared molecular profiles, more precise, targetted treatment . In this way, diagnoses would be made more objective and patients would receive fewer but more effective treatments, leading to improved outcomes and potentially reduced costs. 

Many early examples of this approach come from within cancer, with the targeting of molecular markers specific to certain types of cancer cells [Kummar, S. et al. (2015). Application of molecular profiling in clinical trials for advanced metastatic cancers. Journal of the National Cancer Institute, 107(4), djv003. doi: 10.1093/jnci/djv003]. In a trial where patients were characterised for genetic variants that  influences response to antidepressant and antipsychotic medications, those who were tested received fewer drugs, and they saved an average of \$1,036 in annual prescription costs compared to non-tested patients [Winner. J.G. et al. (2015). Combinatorial pharmacogenomic guidance for psychiatric medications reduces overall pharmacy costs in a 1 year prospective evaluation. Current Medical Research and Opinion. Early online, 23 July 2015. doi:10.1185/03007995.2015.1063483]. The tested patients were also 17\% more likely to keep taking their medications as prescribed.

In both of the above cases, the "ground truth" of the classification was apparent. In more complex and less-understood conditions, where the molecular basis is unclear, sub-typing and identifying clusters can be a formidable problem. It is essentially an unsupervised learning problem, and without careful analysis it is unclear whether any clusters revealed are robust (to changes in parameters and methodology), stable (to the same data gathered from a non-identical set of patients), or meaningful.  Clustering methods can find clusters in very homogeneous data sets, and the assumptions that underlie a given method may hold for some parts of a dataset but not another. Paraphrasing Henning [2007 REF] a dataset may contain clusters that are meaningful, stable and robust but it need not consist only of those clusters. Re-examination of previously identified patient groupings using different methods can increase confidence in the clusters and also suggest new groupings or reveal sub structure within existing clusters. 

For example, asthma is characterised by heterogeneous clinical phenotypes, the  mechanisms involving multiple cellular compartments with a diversity of disease-driving mechanisms. Clustering using clinical features alone has not yielded information on the underlying biology [REF Moore WC, Fitzpatrick AM, Li X, et al. Clinical heterogeneity in the severe asthma research program. Ann Am Thorac Soc 2013; 10: Suppl., S118–S124.] There have been several applications of subtyping using molecular and clinical data. Kuo et al.  (Kuo et al., 2017) analysed sputum transcriptomics data from the U-BIOPRED study [RE] and using a hierarchical clustering approach identified three transcriptional associated clusters (TACs): TAC1 (30 patients) with the highest eosinophilia, TAC2 (22 patients) with the highest sputum neutrophil count and TAC3 (52 patients) with normal to high eosinophilia and containing the lowest fraction of patients with severe asthma. 

Here we re-examine the clusters of Kuo et al. using a variety of approaches (two popular machine learning approaches and (a) stochastic neighbour embedding, (ii) topological data analysis). not only to better understand the robustness and meaningfulness of the Kuo clusters (whether they can be whether they can be reproducibly identified) but also to use the dataset as an exemplar for examining the issue of subtyping and search for possible substructure. 

\section*{Methodology}

We first used an unsupervised approach (t-distributed stochastic neighbour embedding, tSNE (L.J.P. van der Maaten and G.E. Hinton, 2008) (  Bushati et al., 2011)) on the transcriptomics data matrix from Kuo at al., which had been filtered on genes differentially expressed between eosinophil and non-eosinophil associated asthma (of size 104 x 508). We cluster the patients in the lower dimensional space using spectral clustering with k=4. We also cluster the data matrix directly with a new method that does not require specification of the cluster number, Progeny Clustering (Hu et al., 2015)

(why these methods? see next text)

There is a a superfluity of clustering methods with no clear winner for all datasets [REF Olatz Arbelaitz, Ibai Gurrutxaga, Javier Muguerza, Jesús M. Pérez, Iñigo Perona, An extensive comparative study of cluster validity indices, In Pattern Recognition, Volume 46, Issue 1, 2013, Pages 243-256, ISSN 0031-3203, https://doi.org/10.1016/j.patcog.2012.07.021.
(http://www.sciencedirect.com/science/article/pii/S003132031200338X))]


\section*{Results}

\subsection{Clusters}

Figure 1a shows the low dimensional representation of tSNE representation of sputum transcriptomics data with the eosinophil status (as defined in Kuo et al) shown by the colour of the nodes (patients). Figure 1b shows the patients colour-coded according to the cluster assignment from Kuo et al obtained by hierarchical clustering. Interestingly 3 patients appear to be misclassified by Kuo according to the tSNE embedding, all being assigned to TAC1, namely A 220, [5.145679 0.620944], A 256 TAC1 [3.960727 1.273015] and A 619 TAC1 [3.754373 0.430054]. Manual inspection of the tSNE visualisation suggests that in the lower dimensional space, these three patients appear to be closer to patients in TACs 2 or 3.  

Figure 1: The low dimensional embedding from tSNE (run with default parameters) colour coded according to (a) patient eosinophil status as defined by Kuo et al. (b) assignment of patients to three transcriptome-associated clusters (TACs) from Kuo et al. using hierarchical clustering. 

The output of tSNE lends itself to visual inspection and the patients can also be clustered in this lower dimensional space with standard clustering algorithms. Here the data is clustered in the low dimensional space generated by tSNE, using spectral clustering and setting the number of clusters as 4 (Figure 2a). Separately we have clustered the original transcriptome data matrix using ProgenyClust a method that attempts to find the ideal number of clusters. The ProgenyClust program suggest 4 clusters. In Figure 2b the patients represented in the tSNE embedding are colour coded according to cluster assignment by ProgenyClust. 


Figure 2 The low dimensional embedding from tSNE colour coded according to (a) Cluster assignments obtained from spectral clustering of points in the low dimensional space with k set as 4 (b) According to ProgenyClust algorithm applied to the original transcriptome data matrix from Kuo et al. that suggests an ideal number of clusters (in this case 4). The two representations are broadly consistent though not identical.

Similarity to Kuo et al. TACs

An appropriate metric to usefully compare the results of clustering, so as to show how clusters agree or disagree. While there are a cornucopia of possible metrics, ideally several properties should be satisfied: 

\begin{enumerate}
\item The metric should be agnostic as to the labelling used in different partitions. That is, if two schemes group the same set of points into a single cluster but assign it different names, this should not effect the metric value.
\item The metric should allow for partition similarity being achieved by chance
\item The metric should allow for subclusters and partitions that are compatible but non-identical. For example, if a scheme A produces a cluster that is split into two clusters in scheme B, then B is fully compatible (can be used to fully determine) the clustering in A. This implies that the metric is asymmetrical (directional).
\item The metric values should be intuitive. There should be reasonable minimum and maximum values.
\end{enumerate}

To this end, we use two metrics. The first, adjusted mutual information [REF] satisfies all these properties but the third. The second Clustering Compatibility by Conditional Entropy (C3E) is a user-friendly form of an established metric, we shall derive below.

Conditional entropy of Y upon X (H(Y|X)) gives the amount of additional information needed to describe Y given the value of X is known. Thus a vector of the partitioning under scheme Y can be compared against X. The result ranges from 0 (where X completely informs Y) upwards. 

To convert this to a consistent form, we define C3E as:

c3E = 1 – CE / Cemr

Where CE is the conditional entropy of Y upon X (H(Y|X) and Cemr is  the conditional entropy where partition assignment is randomised in either vector and thus forms an approximation of maximum conditional entropy, i.e. where a partition provides no information on the other. In the below analyses, the randomisation is performed 100 times and the mean taken.

C3E will range from around zero  - where partitions mismatch - up to near 1 where the clusters of X will match those in Y or split them.

INSERT TABLE

In summary, the  Progeny and tSNE partitioning are much closer than other pairings (i.e. those that involve Kuo). In turn, they are more informative for Kuo than the reverse.

The distribution of patients in the low dimensional representation generated by application of tSNE suggests a clustering very similar to that of Kuo et al. (refer to Figure 1b). TAC1 is similar to cluster C1, TAC2 maps to C3, and TAC3 maps to C2 and C4. This suggests substructure associated with TAC3. 

Eosinophil distribution across clusters 

We colour code the tSNE representation with the percent eosinophil levels and we see that cluster 1 has a higher percent eosinophil that the others (Figure 3). This is consistent with Kuo et al.

Figure 3: (a) Heatmap spectrum of Eosinophil percent level across the clusters. (b) Boxplot of Eosinophil percent across the clusters 

Neutrophil distribution across clusters  

We colour code the tSNE representation with the percent eosinophil levels (Figure 4) and we see that cluster 2 has a higher percent neutrophils that the others (Figure 3). Cluster C4 has the highest level of Pct??? Neutrophils consistent with TAC2 of Kuo et al. 

The neutrophil distribution suggest that the sub-division of Kuo’s TAC3 into two clusters C2 and C4 may be meaningful with C2 having a higher Pct Neutrophil than C4.
% * <p.agapow@imperial.ac.uk> 2017-11-16T11:29:02.495Z:
% 
% > The Neutrophil distribution suggest that the sub-division of Kuo’s TAC3 into two clusters C2 and C4 may be meaningful with C2 having a higher Pct Neutrophil than C4.
% I like this – an emergent property that shows clusters might be real
% 
% ^.
Figure 4: (a) Heatmap spectrum of Neutrophil percent level across the clusters. (b) C3 (TAC2) has the highest Pct Neutrophil and we also see the difference between C2 and C4 possibly reflecting the substructure of T3. 

Distribution of U-BIOPRED cohorts across the clusters 
Each of the patients has an associated U-BIOPRED cohort assignment, We next map the U-BIOPRED asthma cohorts (a-severe, b-severe smokers, c-mild to moderate) to the tSNE representation (Figure 5). 

Figure 5: This again reflects some substructure in TAC3, (C2, C4) with the ‘upper’ cluster (C2) having more mild to moderate patients. 

Genes that discriminate between C2 and C4?

Kuo et al note that some expression levels of some genes characterize TACs, for example CCR3 for TAC1. We observe this also (Figure 6 (a), and we seek to explore if signatures exist that discriminate between C2 and C4 that is the two sub-clusters of TAC3 that we have suggested. KCNJ15 is a possibility showing some differences between C2 and C4 (refer to Figure 6(b)).

Figure 6 (a) Variation of CCR3 across the 4 clusters, C1 (TAC1) shows elevated levels (b) variation of KCNJ15 shows elevated levels for C3 (TAC2) and some differences between C2, C4 i.e. the sub clusters in TAC3.

TDA Analysis (Ping’s work)

We applied Topological Data Analysis (TDA) as implemented in the Ayasdi software. The results showed one TAC1 cluster, one TAC2 cluster and three TAC3 subclusters TAC3a, TAC3b and TAC3c (Figure 7). 
This is similar to the tSNE approach except that TAC3 has 3 rather than 2 sub-clusters. 
For sputum eosinophils, TAC1 had the highest eosinophils, consistent with tSNE and Kuo et al. TAC2 had low to medium eosinophils and three TAC3 clusters showed heterogeneous results (Figure 1B). TAC2 also showed the highest neutrophils, consistent with tSNE and Kuo et al., while TAC1 and TAC3 clusters had heterogeneity of neutrophils. It was also observed that the TAC3b patients were younger than TAC3a patients.


\bibliography{tacs}

\end{document}